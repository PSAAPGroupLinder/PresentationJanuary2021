%!TEX root = ../presentation.tex
\begin{frame}{Gradient-Enhanced Damage-Plasticity Framework}
    \textbf{Introducing \emph{nonlocality} of damage by means of the implicit gradient approach:}
    \begin{minipage}{0.54\textwidth}
    \begin{align*}
        \sig &=  (1-\omega) \,\bar\sig  \\
             &= (1-\omega)  \, \Cel : \eps^\*{el} \\
        \eps^\*{el} &= \eps - \eps^\*{p}\\
        \omega &= \omega ( \alphaD, \textcolor{red}{{\tilde{\kappa}}} ) \\
        \dot \kappa &=  \dot \kappa (\dot \eps^\*{p}, \sig)
    \end{align*}
    \begin{align*}
        \nabla \sig + \boldsymbol f &= \boldsymbol 0 \\
    \textcolor{red}{\alphaDNL - l^2 \, \nabla ^2 \alphaDNL}&= \textcolor{red}{\alphaD}
    \end{align*}
    \end{minipage}
    \begin{minipage}{0.44\textwidth}
        \vspace{6.0em}

        Damage $\omega$ related to local $\alphaD$ and  nonlocal $\textcolor{red}{{\tilde{\kappa}}}$ \\[3.5em]
    Implicit \emph{Helmholtz-like} equation for the nonlocal damage driving field
    \end{minipage}

    \textbf{Characteristics:}
    \begin{itemize}
        \item Strongly nonlocal approach
        \item Material length scale introduced into the continuum formulation by means of parameter $l$
        \item Different formulations (nonlocal, over-nonlocal, ...)
        \item Fully coupled system of PDEs
    \end{itemize}
\end{frame}
