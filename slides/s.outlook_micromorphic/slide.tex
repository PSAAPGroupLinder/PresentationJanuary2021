%!TEX root = ../presentation.tex

\begin{frame}{}
    \textbf{Combination of the micromorphic continuum and the gradient-enhanced damage framework:}
    \begin{itemize}
        \item Incorporation of the micromorphic continuum (in the sense of A.C. Eringen) into the continuum damage framework:
                \begin{align*}
                    \sigma^{kl} &= (1-\omega) \, \bar\sigma^{kl} \\
                    s^{kl} &= (1-\omega) \, \bar{s}^{kl} \\
                    m^{klm} &= (1-\omega) \, \bar{m}^{klm} \\
                \end{align*}
        \item Damage parameter $\omega$ related to internal state variable(s) $\boldsymbol \kappa$ (local) and $\boldsymbol {\tilde\kappa}$ (nonlocal):
                \begin{equation*}
                    \omega = \omega (  \boldsymbol \kappa, \boldsymbol {\tilde\kappa} )
                \end{equation*}
        \item Evolution of $\boldsymbol \kappa$ related to the evolution of plastic deformations:
            \begin{equation*}
                \boldsymbol {\dot \kappa } =  \boldsymbol {\dot \kappa } ( \boldsymbol {\dot F}^\mathrm{p}, \boldsymbol {\dot \chi}^\mathrm{p}, \nabla \boldsymbol {\dot \chi}^\mathrm{p},%
                \boldsymbol \sigma, \boldsymbol s, \boldsymbol {m})
            \end{equation*}
        \item Separate PDE for representing the nonlocal character of damage:
            \begin{equation*}
                {\boldsymbol {\tilde \kappa} - l^2 \, \nabla ^2 \boldsymbol {\tilde\kappa} }= \boldsymbol \kappa
            \end{equation*}
        \item Naturally incorporates the micromorphic continuum into the framework of continuum damage mechanics
        \item Advantage: Separate length scale for damage, natural length scales introduced by the micromorphic continuum remain \emph{available} for multiscale modeling
    \end{itemize}
\end{frame}
