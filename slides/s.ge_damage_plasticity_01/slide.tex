%!TEX root = ../presentation.tex
\begin{frame}{Coupled Damage-Plasticity Models}
    \textbf{Damage-plasticity models, classical local form (small strain regime):}
    \begin{minipage}{0.54\textwidth}
    \begin{align*}
        \sig &=  (1-\omega) \,\bar\sig  \\
             &= (1-\omega)  \, \Cel : \eps^\*{el} \\
        \eps^\*{el} &= \eps - \eps^\*{p}\\
        \omega &= \omega ( \alphaD ) \\
        \dot \kappa &=  \dot \kappa (\dot \eps^\*{p}, \sig)
    \end{align*}
            
    \begin{align*}
        \nabla \sig + \boldsymbol f &= \boldsymbol 0 \\
    \end{align*}
    \end{minipage}
    \begin{minipage}{0.44\textwidth}
        Relation between nominal stress ($\sig$) and effective stress ($\bar \sig$) via damage parameter $\omega$ \\[.5em]
        Decomposition of the total strain $\eps$ into elastic and plastic parts \\[.5em]
        Evolution laws for damage $\omega$ and damage driving internal state variables $\alphaD$  \\[2.0em]
    Classical equilibrium equation
    \vspace{0.0em}
    \end{minipage}

    \begin{itemize}
        \item Inelastic (plastic) deformations
        \item Degradation of stiffness due to material damage
        \item Representation of hardening and softening material behavior of quasi-brittle materials:
        \begin{itemize}
            \item Ductile behavior in confined compression
            \item Brittle behavior in tension
        \end{itemize}
        \item Numerous applications to various cohesive-frictional materials (concrete, rock mass, ...)
        \item Different approaches for regularizing softening material available in the literature
    \end{itemize}
\end{frame}
