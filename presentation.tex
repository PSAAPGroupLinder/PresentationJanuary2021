% USE THIS TO CREATE A PRESENTATION
\documentclass[8pt,a4,t]{beamer}

\usepackage{amsfonts,amssymb,amsmath,exscale} 
% \usepackage[final]{animate}
\usepackage{pst-node}
\usepackage[english]{babel}
\usepackage{colordvi}
\usepackage{psfrag}
\usepackage{macros}


\mode<presentation>{\usetheme{stanford}}

\def\supertiny{\font\supertinyfont = cmss10 at 3.75pt \relax \supertinyfont}

%-----------------------------------------------------------------------
% lecture title
\renewcommand{\mytitle}{3D modeling of fracture/discontinuities for classical and micromorphic continua}
% project authors
\renewcommand{\myauthor}{Tim Ngo, Prajwal Kammardi Arunachala, Matthias Neuner, Christian Linder}
% institute
\renewcommand{\myinstitute}{Stanford University $\cdot$ Department of Civil \& Environmental Engineering}
% conference name
\renewcommand{\myconference}{PSAAP Presentation - January 7, 2021}
% footline
\newcommand{\footlinetextleft}{Christian Linder}
\newcommand{\footlinetextcenter}{Stanford University}
\newcommand{\footlinetextright}{PSAAP-Presentation}

%-----------------------------------------------------------------------
% begin of document
%-----------------------------------------------------------------------
\newcommand{\PartNr}{}
\newcommand{\PartShort}{}

\begin{document}

{\small
% create title slide
\titleslide

\load{slides/s.01}				% slide 01
\load{slides/s.02}				% slide 02
\load{slides/s.03}				% slide 03
\load{slides/s.04}				% slide 04


\newcommand{\sig}{\boldsymbol {\sigma}}
\newcommand{\alphaDNL}{\tilde{\kappa}}
\newcommand{\alphaD}{\kappa}
\renewcommand{\*}{\mathrm}
\renewcommand{\eps}{\boldsymbol \varepsilon}
\newcommand{\Cel}{\mathbb{C}}
\load{slides/s.ge_damage_plasticity_01}				
\load{slides/s.ge_damage_plasticity_02}			
\load{slides/s.l_shape_01}				
\load{slides/s.l_shape_02}			
\load{slides/s.mode_ii_01}		
\load{slides/s.mode_ii_02}	
\load{slides/s.triax_01}
\load{slides/s.triax_02}
\load{slides/s.outlook_micromorphic}
}
%
%-----------------------------------------------------------------------
% end of document
%-----------------------------------------------------------------------
\end{document}
